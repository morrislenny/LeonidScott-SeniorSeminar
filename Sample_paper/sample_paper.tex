% This is a sample document using the University of Minnesota, Morris, Computer Science
% Senior Seminar modification of the ACM sig-alternate style. Much of this content is taken
% directly from the ACM sample document illustrating the use of the sig-alternate class. Certain
% parts that we never use have been removed to simplify the example, and a few additional
% components have been added.

% See https://github.com/UMM-CSci/Senior_seminar_templates for more info and to make
% suggestions and corrections.

\documentclass{sig-alternate}
\usepackage{color}
\usepackage[colorinlistoftodos]{todonotes}

%%%%% Uncomment the following line and comment out the previous one
%%%%% to remove all comments
%%%%% NOTE: comments still occupy a line even if invisible;
%%%%% Don't write them as a separate paragraph
%\newcommand{\mycomment}[1]{}

\begin{document}

% --- Author Metadata here ---
%%% REMEMBER TO CHANGE THE SEMESTER AND YEAR AS NEEDED
\conferenceinfo{UMM CSci Senior Seminar Conference, April 2019}{Morris, MN}

\title{Wing Design Using SAIL}

\numberofauthors{1}

\author{
% The command \alignauthor (no curly braces needed) should
% precede each author name, affiliation/snail-mail address and
% e-mail address. Additionally, tag each line of
% affiliation/address with \affaddr, and tag the
% e-mail address with \email.
\alignauthor
Leonid Scott\\
	\affaddr{Division of Science and Mathematics}\\
	\affaddr{University of Minnesota, Morris}\\
	\affaddr{Morris, Minnesota, USA 56267}\\
	\email{scot0530@morris.umn.edu}
}

\maketitle
\begin{abstract}
This paper takes a deep dive into Gaier et al's paper, \textit{Data-Efficient Design Exploration through Surrogate-Assisted Illumination}.
While many evolutionary algorithms attempt to find the optimal solution to a problem space, Gaier et al develops an evolutionary algorithm that \textit{illuminates} the problem space. 
This concept is particularly attractive in aerospace engineering as it allows engineers to properly survey the possible landscape of wing designs before investing into one particular concept heavily.

Gaier et al build off of the work of a previous algorithm called \textit{MAP-Elites}. 
MAP-Elites is designed to provide a number of high performing solutions that represent diverse regions of performance from the problem space. 
Experiments in Monet et al show MAP-Elites to be effective in creating this diverse set of solutions.
However, MAP-Elites has been shown to be extremely computationally expensive.
Moreover, MAP-Elites relies on a static model that does not change during an evolutionary run.

Gaier et al set out to improve MAP-Elites in the context of problem spaces where:
 \begin{enumerate}
   \item Running a high fidelity model is too computationally expensive to run per individual created.
   \item Running each generation comes at a significant computational cost and accuracy cost.
 \end{enumerate}
Gaier et al's algorithm, SAIL (Surrogate Assisted Illumination), uses a surrogate model to approximate the fitness of a solution instead of high fidelity model.
In its current implementation, SAIL uses a Gaussian Process as a surrogate model. In order to guess where to select new individuals, SAIL uses Bayesian Optimization (BO).
BO allows SAIL to reduce the number of generations it needs to acquire a strong, diverse set of solutions that represent a problem space well. 

This paper conducts two different experiments to test SAIL's viability in the context of aerospace engineering.
The first test attempts to design a set of two dimensional airfoils. A second test was added to design the three dimensional shape of a velomobile.
This second test was deemed important as the model to test it was much more complicated than its two dimensional counterpart.
This difficult model would stress SAIL's Surrogate Assisted model to a greater extent.

\end{abstract}

\keywords{Evolutionary Computation, MAP-Elites, Gausian Process, Bayesian Optimization}

\section{Introduction}
\label{sec:introduction}



\section{SAIL Build-up}
\label{sec:body}

\subsection{Evolutionary Algorithms}
\label{sec:typeChangesSpecialChars}

\subsection{SAIL Introduction}
\label{sec:mathEquations}

\subsection{MAP-Elites}
\label{sec:displayEquations}

\section{Conclusions}
\label{sec:conclusions}

This paragraph will end the body of this sample document.
Remember that you might still have Acknowledgments or
Appendices; brief samples of these
follow.  There is still the Bibliography to deal with; and
we will make a disclaimer about that here: with the exception
of the reference to the \LaTeX\ book, the citations in
this paper are to articles which have nothing to
do with the present subject and are used as
examples only.

\section*{Acknowledgments}
\label{sec:acknowledgments}

This section is optional; it is a location for you
to acknowledge grants, funding, editing assistance and
what have you.

You want to use the \texttt{\textbackslash section*} version of the \texttt{section}
command, as an acknowledgments section typically does \emph{not} get
a number.

It is common (but by no means necessary) for students to thank
their advisor, and possibly other faculty, friends, and family who provided
useful feedback on the paper as it was being written.

In the present case, for example, the
authors would like to thank Gerald Murray of ACM for
his help in codifying this \textit{Author's Guide}
and the \textbf{.cls} and \textbf{.tex} files that it describes.

% The following two commands are all you need in the
% initial runs of your .tex file to
% produce the bibliography for the citations in your paper.
\bibliographystyle{abbrv}
% sample_paper.bib is the name of the BibTex file containing the
% bibliography entries. Note that you *don't* include the .bib ending here.
\bibliography{sample_paper}  
% You must have a proper ".bib" file
%  and remember to run:
% latex bibtex latex latex
% to resolve all references

\end{document}
