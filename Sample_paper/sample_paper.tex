% This is a sample document using the University of Minnesota, Morris, Computer Science
% Senior Seminar modification of the ACM sig-alternate style. Much of this content is taken
% directly from the ACM sample document illustrating the use of the sig-alternate class. Certain
% parts that we never use have been removed to simplify the example, and a few additional
% components have been added.

% See https://github.com/UMM-CSci/Senior_seminar_templates for more info and to make
% suggestions and corrections.

\documentclass{sig-alternate}
\usepackage{color}
\usepackage[colorinlistoftodos]{todonotes}

%%%%% Uncomment the following line and comment out the previous one
%%%%% to remove all comments
%%%%% NOTE: comments still occupy a line even if invisible;
%%%%% Don't write them as a separate paragraph
%\newcommand{\mycomment}[1]{}

\begin{document}

% --- Author Metadata here ---
%%% REMEMBER TO CHANGE THE SEMESTER AND YEAR AS NEEDED
\conferenceinfo{UMM CSci Senior Seminar Conference, April 2019}{Morris, MN}

\title{Wing Design Using SAIL}

\numberofauthors{1}

\author{
% The command \alignauthor (no curly braces needed) should
% precede each author name, affiliation/snail-mail address and
% e-mail address. Additionally, tag each line of
% affiliation/address with \affaddr, and tag the
% e-mail address with \email.
\alignauthor
Leonid Scott\\
	\affaddr{Division of Science and Mathematics}\\
	\affaddr{University of Minnesota, Morris}\\
	\affaddr{Morris, Minnesota, USA 56267}\\
	\email{scot0530@morris.umn.edu}
}

\maketitle
\begin{abstract}
This paper takes a deep dive into Gaier et al's paper, \textit{Data-Efficient Design Exploration through Surrogate-Assisted Illumination}.
While many evolutionary algorithms attempt to find the optimal solution to a problem space, Gaier et al develops an evolutionary algorithm that \textit{illuminates} the problem space. 
This concept is particularly attractive in aerospace engineering as it allows engineers to properly survey the possible landscape of wing designs before investing into one particular concept heavily.

Gaier et al build off of the work of a previous algorithm called \textit{MAP-Elites}. 
MAP-Elites is designed to provide a number of high performing solutions that represent diverse regions of performance from the problem space. 
Experiments in Monet et al show MAP-Elites to be effective in creating this diverse set of solutions.
However, MAP-Elites has been shown to be extremely computationally expensive.
Moreover, MAP-Elites relies on a static model that does not change during an evolutionary run.

Gaier et al set out to improve MAP-Elites in the context of problem spaces where:
 \begin{enumerate}
   \item Running a high fidelity model is too computationally expensive to run per individual created.
   \item Running each generation comes at a significant computational cost and accuracy cost.
 \end{enumerate}
Gaier et al's algorithm, SAIL (Surrogate Assisted Illumination), uses a surrogate model to approximate the fitness of a solution instead of high fidelity model.
In its current implementation, SAIL uses a Gaussian Process as a surrogate model. In order to guess where to select new individuals, SAIL uses Bayesian Optimization (BO).
BO allows SAIL to reduce the number of generations it needs to acquire a strong, diverse set of solutions that represent a problem space well. 

This paper conducts two different experiments to test SAIL's viability in the context of aerospace engineering.
The first test attempts to design a set of two dimensional airfoils. A second test was added to design the three dimensional shape of a velomobile.
This second test was deemed important as the model to test it was much more complicated than its two dimensional counterpart.
This difficult model would stress SAIL's Surrogate Assisted model to a greater extent.

\end{abstract}

\keywords{Evolutionary Computation, MAP-Elites, Gausian Process, Bayesian Optimization}

\section{Introduction}
\label{sec:introduction}

\todo[inline]{Insert better opening statement}

Fluid dynamics stands as one of the most difficult problem spaces to model and thus design in.
The equations that govern fluid flow, known as the \textit{Navier-Stokes Equations}, are a set of of partial differential equations with no known solution.
As a result, aerospace engineers must use extraordinary computational power to approximate navier stokes results.
High fidelity models of aerodynamic devices can take hours to simulate and still impart noticeable error.
Given the difficulty in modeling fluid flow, designing these devices is an extraordinary challenge.

Optimization tools aid in this challenge by helping engineers at the end of design cycle by refining a design to what is known as a local optima.
In difficult problem spaces such as fluid dynamics, there is no guarantee that an optimizer has found the best possible solution across the entire problem space.
However, the optimizer can be very confident that it has found the best solution in one small region.
We call the optimal solution across the problem space a \textit{global optima}, and an optimal solution in a subsection of the problem space a \textit{local optima}. 

In 2014, Autodesk, a producer of modelling and optimizing software found that their software was being used in a peculiar way by engineers.
Instead of using optimizing tools at the end of the design process to refine designs, engineers used them at the beginning to explore the space of possible designs (Bradner et al).
Instead of using these tools to find, with great accuracy, one local optima, they were used to find several throughout the problem space.
By finding several local optima, designers could see what tradeoffs are inherent in the problem space, and hone in on regions of interest.
This process is known as \textit{illumination}.

Over the course of several years, a research group consisting of Adam Gaier, Alexander Asteroth, and Jean-Baptiste Mouret have developed a purpose built algorithm to illuminate problem spaces, particularly when models are computationally expensive.
This algorithm, known as \textit{Surrogate-Assisted Illumination} (SAIL) uses a thus far reliable method for exploring problem spaces known as evolutionary computation, albeit in a very specialized form.
The goal of SAIL is to produce a series of high performing solutions, across the problem space, to challenging engineering problems.

\todo[inline]{Insert stuff about paper structure}

\section{Background: Evolutionary Algorithms}
\label{sec:evolutionaryAlgorithms}

SAIL uses a type of Evolutionary Algorithm to explore and exploit the problem space.
Evolutionary Algorithms (EA's) are stochastic algorithms (algorithms including randomness) inspired by biological evolution.
The premise is that a population of potential solutions is tested against a model of the problem and assigned a \textit{fitness score}.
The individuals with the best fitness scores move onto the next generation and produce ``offspring''.
Over many generations the performance against the model will improve until either a target performance is obtained, or the algorithm reaches a fixed number of generations. 

\todo[inline]{Might include blurb about exploration vs exploitation? Will be decided after MAP-Elites section is done}

\section{SAIL Build-up}
\label{sec:SAILBuildUp}
SAIL is made up of several components. There is:
\begin{itemize}
  \item \textbf{MAP Elites:} This is the evolutionary algorithm that SAIL is based on.
  \item \textbf{Bayesian Optimization:} The mechanism by which SAIL decides where and how new individuals will be created.
  \item \textbf{Gaussian Processes:} A way of not having to compute the expensive model by approximating it in what is known as a surrogate model.
\end{itemize}
In this section, we will go through each of these components in depth before bringing them all together and constructing Surrogate Assisted Illumination.

\subsection{MAP-Elites}
\label{sec:MAP-Elites}

MAP Elites is an Evolutionary Algorithm developed in April 2015 by Jean-Baptiste Mouret and Jeff Clune.
The purpose of MAP Elites is to \textit{illuminate} the problem space.
That is, to produce a series of high performing solutions that represent different trade offs and insights into the problem space.
Before moving into the details of MAP Elites, it is important to understand the terminology surrounding individuals in MAP Elites.

\subsubsection{Genotypes}
\label{sec:genotypes}

An important aspect in the application of any Evolutionary Algorithm is the way that individuals are represented.
In engineering contexts, there is a need to represent a physical object in all of its complexities in a compact and understandable form.
This representation is referred to an \textit{encoding} or a \textit{genotype} interchangeably.
Running with the theme of aerospace engineering, a common way to represent a two dimensional \textit{foil} (wing shape) is with three numbers.
These three numbers refer to important geometric features about the foil shape, but also directly relate to foil performance.
We will call these values M, P, and T. In very rough terms, these values refer to:
\begin{itemize}
  \item M refers to how ``arched'' the foil is is.
  \item P refers to where the most ``arched'' part of the foil exists.
  \item T refers to the thickness of the foil at the most arched section.
\end{itemize}
With this compact genotype for a 2D foil, it is possible to represent a large range of complex shapes, with only three numbers.

\subsubsection{Phenotypes}
\label{sec:phenotypes}

Once we have a representation of an individual, we can start deriving its \textit{behaviors}, or \textit{phenotypes}.
In this example, those behaviors might include the Lift and Drag of the foil in a certain condition as well as its Cross-Sectional Area.
In this case, the \textit{Feature Space}, the space of all possible phenotypes, will exist in three dimensions: Lift, Drag, and Cross-Sectional Area.
The function that takes in an individual and returns its phenotype is known as the \textit{behavior function}, denoted as $b(x)$.

\subsubsection{Fitness}
\label{sec:fitness}

MAP Elites requires some sort of specific score so that it can strictly tell that one foil is ``better'' than another.
The \textit{fitness function} $f(x)$ takes an individual and its behaviors and returns a score quantifying how well it accomplishes our specified goals.
When developing a foil shape, we most certainly care about Lift and Drag, but for weight and structural reasons, we might also care about the foil’s cross-sectional area.
A fitness function that encompases these behaviors into a single score might look like this:
$$f(x) = a*Lift(x) - b *Drag(x) + c*Area(x)$$
Where $a$, $b$, and $c$ are constants defined by the engineer based on which factors are more important than others.
In this example, $f(x)$ is set setup such that a \textit{higher} fitness score means a foil is better at achieving our goals, but it doesn't have to be that way.
Fitness functions can be setup such that a good score is small score, a small absolute value, etc...

\subsubsection{MAP Elites}
\label{MAPElitesSub}

\subsection{Bayesian Optimization}
\label{bayesianOptimization}

\subsection{Gaussian Process}
\subsubsection{gaussianProcess}

\subsection{SAIL Algorithm}
\label{SAILAlgorithm}

\section{Experiments}
\label{experiments}

\subsection{2D Foil Experiment}
\label{2DFoilExperiment}

\subsection{3D Foil Experiment: Velomobile Experiment}
\label{3DFoilExperiment}

\section{Conclusions}
\label{sec:conclusions}

This paragraph will end the body of this sample document.
Remember that you might still have Acknowledgments or
Appendices; brief samples of these
follow.  There is still the Bibliography to deal with; and
we will make a disclaimer about that here: with the exception
of the reference to the \LaTeX\ book, the citations in
this paper are to articles which have nothing to
do with the present subject and are used as
examples only.

\section*{Acknowledgments}
\label{sec:acknowledgments}

This section is optional; it is a location for you
to acknowledge grants, funding, editing assistance and
what have you.

You want to use the \texttt{\textbackslash section*} version of the \texttt{section}
command, as an acknowledgments section typically does \emph{not} get
a number.

It is common (but by no means necessary) for students to thank
their advisor, and possibly other faculty, friends, and family who provided
useful feedback on the paper as it was being written.

In the present case, for example, the
authors would like to thank Gerald Murray of ACM for
his help in codifying this \textit{Author's Guide}
and the \textbf{.cls} and \textbf{.tex} files that it describes.

% The following two commands are all you need in the
% initial runs of your .tex file to
% produce the bibliography for the citations in your paper.
\bibliographystyle{abbrv}
% sample_paper.bib is the name of the BibTex file containing the
% bibliography entries. Note that you *don't* include the .bib ending here.
\bibliography{sample_paper}  
% You must have a proper ".bib" file
%  and remember to run:
% latex bibtex latex latex
% to resolve all references

\end{document}
